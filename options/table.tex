% Created 2021-06-22 Tue 12:43
% Intended LaTeX compiler: xelatex
\documentclass[9pt]{article}
\usepackage{graphicx}
\usepackage{grffile}
\usepackage{longtable}
\usepackage{wrapfig}
\usepackage{rotating}
\usepackage[normalem]{ulem}
\usepackage{amsmath}
\usepackage{textcomp}
\usepackage{amssymb}
\usepackage{capt-of}
\usepackage[hidelinks]{hyperref}
\usepackage[a4paper, total={6in,9in}]{geometry}
\usepackage{booktabs}
\usepackage{minted}
\usepackage{parskip}
\usepackage{sectsty}
\sectionfont{\fontsize{12}{15}\selectfont}
\subsectionfont{\fontsize{11}{11}\selectfont}
\setlength\parindent{0pt}
\usepackage{parskip}
\usepackage{pifont}
\makeatletter
\def\@makechapterhead#1{%
{\parindent \z@ \raggedright \normalfont
\ifnum \c@secnumdepth >\m@ne
\LARGE\bfseries \thechapter~
\fi
\interlinepenalty\@M
\LARGE \bfseries #1\par\nobreak
\vskip 10\p@
}}
\def\@makeschapterhead#1{%
{\parindent \z@ \raggedright
\normalfont
\interlinepenalty\@M
\Huge \bfseries  #1\par\nobreak
\vskip 10\p@
}}
\makeatother
\author{LegalTurf Website}
\date{June 19, 2021}
\title{Website Creation Options}
\hypersetup{
 pdfauthor={LegalTurf Website},
 pdftitle={Website Creation Options},
 pdfkeywords={},
 pdfsubject={},
 pdfcreator={Emacs 25.2.2 (Org mode 9.3.6)},
 pdflang={English}}
\begin{document}

\maketitle

\section{Mainstream Blogging Options}
\label{sec:org62b9c73}

\begin{itemize}
\item There are three ways of creating the required website: using
Blogger, Wordpress, and Ghost. Unfortunately, \emph{none} of these are
static, which means they cannot be hosted on GitHub Pages.
\item As such, prior to our implementation of the website, it must
\emph{first} be hosted on a webserver which supports dynamic webpages.
\item It is also a good idea to host first since domain names tend to
be ``squatted'' when interest is shown.
\item This table assumes your billing cycle is annual to minimize
monthly cost, since buying on a per-month basis is far more
expensive.
\item Each of Blogger, Wordpress, and Ghost provide a certain feature
set, and are priced somewhat differently.
\item This table compares these by assigning a difficulty score for the
implementation of any given feature, with 0 being built-in, and
10 being impossible. (Lower is better).
\end{itemize}

\begin{center}
\begin{tabular}{lccc}
\toprule
 & \textbf{Blogger} & \textbf{Wordpress} & \textbf{Ghost}\\
 & \small Default hosting cost & \small \$13/mo & \small \$9/mo\\
\midrule
Blogging & 6 & 5 & 3\\
Other content & 8 & 6 & 7\\
SEO & 10 & 10 & 0\\
Membership \& email & 9 & +\$16/mo & 3\\
Email newsletters & 9 & +\$16/mo & 3\\
Subscribers & 10 & 10 & 3\\
SSL & 9 & 7 & 4\\
\midrule
Content Creation & 7 & 5 & 2\\
Website editing (GUI) & 10 & 8 & 5\\
Support & 9 & 8 & 6\\
Sending emails & Manual & 6 & 3\\
\bottomrule
\end{tabular}
\end{center}

\begin{itemize}
\item The final rows deal with the ease of use and maintenance on the
user's (your) end, which should give an idea of how much support
will be later required.
\item Note that sending emails \emph{without} Wordpress, Ghost, or MailChimp
almost assures that they'll be trapped by a spam filter and never
actually read by a human.
\end{itemize}

As such, \textbf{Ghost} is the most cost-effective and fully-featured option
of the three. Blogger is nearly obsolete and difficult to use.
Wordpress is the most customizable, but requires a lot of work and
money to run smoothly.


\section{GitHub Pages: Zero Cost, Static}
\label{sec:org08ccc5f}

There is one more, possibly zero-cost option. GitHub provides a
hosting page for free. However, the domain must end in \texttt{.github.io},
and has a certain technical learning curve (which can be taught in a
one hour session). Emails will have to be
sent manually (\texttt{cc}), and email data also collected manually, since
there's no server-side database support.

However, the advantage of GitHub pages is the \textbf{zero upfront cost,
forever}, and the possibility to eventually shift to a non-GitHub
domain without having to stop using Pages.

\section{Get Back to Us!}
\label{sec:orgddd991d}

Let us know which of the four (3 mainstream or GitHub) options you
want to move forward with, and we'll coordinate from there.
\end{document}